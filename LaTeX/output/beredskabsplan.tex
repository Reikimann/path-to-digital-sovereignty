\documentclass[a4paper,11pt]{book}

% Required packages
\usepackage[utf8]{inputenc}
\usepackage[T1]{fontenc}
\usepackage[danish]{babel}  % Danish language support
\usepackage{lmodern}        % Modern Latin font
\usepackage{geometry}       % Page geometry
\usepackage{graphicx}       % Images
\usepackage{hyperref}       % Hyperlinks
\usepackage{xcolor}         % Colors
\usepackage{titlesec}       % Title formatting
\usepackage{enumitem}       % List customization
\usepackage{fancyhdr}       % Headers and footers
\usepackage{tcolorbox}      % Colored boxes
\usepackage{listings}       % Code listings
\usepackage{microtype}      % Typography improvements
\usepackage{booktabs}       % Better tables

% Page geometry
\geometry{
    a4paper,
    top=2.5cm,
    bottom=2.5cm,
    left=3cm,
    right=3cm,
    headheight=14pt
}

% Colors
\definecolor{primary}{RGB}{0, 60, 120}   % Dark blue
\definecolor{secondary}{RGB}{0, 110, 180} % Medium blue
\definecolor{accent}{RGB}{0, 150, 200}    % Light blue
\definecolor{light}{RGB}{240, 245, 250}   % Very light blue
\definecolor{dark}{RGB}{50, 50, 50}       % Dark gray

% Hyperlink styling
\hypersetup{
    colorlinks=true,
    linkcolor=secondary,
    urlcolor=accent,
    citecolor=secondary,
    pdfauthor={IT-Beredskab},
    pdftitle={IT-Beredskabsplan},
    pdfsubject={Beredskabsplan for IT-systemer},
    pdfkeywords={IT, beredskab, krise, planlægning}
}

% Title formatting
% Modified title formatting for chapters (without "Chapter X")
\titleformat{\chapter}[display]
{\normalfont\huge\bfseries\color{primary}}
{}{0pt}{\Huge}

\titlespacing*{\chapter}{0pt}{0pt}{40pt}

\titleformat{\section}
{\normalfont\Large\bfseries\color{secondary}}
{\thesection}{1em}{}
\titlespacing*{\section}{0pt}{3.5ex plus 1ex minus .2ex}{2.3ex plus .2ex}

\titleformat{\subsection}
{\normalfont\large\bfseries\color{secondary}}
{\thesubsection}{1em}{}

% Headers and footers
\pagestyle{fancy}
\fancyhf{}
\fancyhead[LE,RO]{\thepage}
\fancyhead[RE]{\textit{\nouppercase{\leftmark}}}
\fancyhead[LO]{\textit{\nouppercase{\rightmark}}}
\renewcommand{\headrulewidth}{0.4pt}
\renewcommand{\footrulewidth}{0pt}

% Box for important notes
\newtcolorbox{infobox}{
    colback=light,
    colframe=accent,
    boxrule=0.5pt,
    arc=3mm,
    fonttitle=\bfseries,
    top=3mm,
    bottom=3mm,
    left=3mm,
    right=3mm
}

% Code listing style
\lstset{
    basicstyle=\ttfamily\small,
    backgroundcolor=\color{light},
    frame=single,
    framesep=3pt,
    breaklines=true,
    postbreak=\mbox{\textcolor{accent}{$\hookrightarrow$}\space},
    numbers=left,
    numberstyle=\tiny\color{dark},
    numbersep=5pt,
    showspaces=false,
    showstringspaces=false,
    showtabs=false,
    tabsize=2,
    captionpos=b,
    keywordstyle=\color{secondary}\bfseries,
    commentstyle=\color{dark}\itshape,
    stringstyle=\color{accent}
}

% Begin document
\begin{document}

% Title page
\begin{titlepage}
    \centering
    \vspace*{2cm}
    {\Huge\bfseries IT-Beredskabsplan\par}
    \vspace{1cm}
    {\Large En planlægningsmetode ved tab af adgang til IT-systemer\par}
    \vspace{2cm}
    
    \vspace{2cm}
    {\Large\itshape En fem-faset plan for krisehåndtering\par}
    \vfill
    
    % Bottom of the page
    {\large \today\par}
\end{titlepage}

% Table of contents
\tableofcontents
\newpage

% Main content - will be replaced by the Python script
\chapter{Faser i Beredskabsplanen}\label{faser-i-beredskabsplanen}

For bedre at kunne afstemme indsatsen med trusselsbilledet arbejder vi
med en fem-faset model. Hver fase har sit eget fokus, tilpasset
trusselsniveauet og organisationens aktuelle behov:

\begin{center}\rule{0.5\linewidth}{0.5pt}\end{center}

\section{Fase 1: Ingen handling
nødvendig}\label{fase-1-ingen-handling-nuxf8dvendig}

\begin{itemize}
\tightlist
\item
  Truslen vurderes som så usandsynlig, at ingen forberedelse er
  nødvendig.
\item
  Fokus på opmærksomhed, men ingen tekniske eller organisatoriske tiltag
  igangsættes.
\end{itemize}

\begin{center}\rule{0.5\linewidth}{0.5pt}\end{center}

\section{Fase 2: Planlægningsfase}\label{fase-2-planluxe6gningsfase}

\begin{itemize}
\tightlist
\item
  Truslen er stadig usandsynlig, men tilstrækkelig til at
  beredskabsplaner udarbejdes.
\item
  Overblik over afhængigheder, systemer og nødvendige tilpasninger
  dokumenteres.
\item
  Projektet offentliggøres muligvis i open source-format for at styrke
  samarbejde og samfundsberedskab.
\end{itemize}

\begin{center}\rule{0.5\linewidth}{0.5pt}\end{center}

\section{Fase 3: Blød overgang}\label{fase-3-bluxf8d-overgang}

\begin{itemize}
\item
  Truslen er sandsynlig nok til, at man begynder en gradvis transition.
\item
  Fokus på balancering af økonomi og organisatorisk forandringsledelse.
\item
  Pilottests igangsættes, fx med:

  \begin{itemize}
  \tightlist
  \item
    Linux på klientmaskiner
  \item
    Alternativ kontorpakke (LibreOffice, OnlyOffice, Collabora)
  \item
    Lokal e-mailserver
  \item
    Udskiftning af cloud-tjenester med lokale alternativer
  \end{itemize}
\end{itemize}

\begin{center}\rule{0.5\linewidth}{0.5pt}\end{center}

\section{Fase 4: Akut overgang}\label{fase-4-akut-overgang}

\begin{itemize}
\tightlist
\item
  Truslen vurderes som umiddelbar og troværdig.
\item
  Alle ressourcer mobiliseres til hurtig implementering af planlagte
  løsninger.
\item
  Kritiske tjenester migreres under tidspres for at sikre fortsat drift.
\item
  Nødprocedurer for områder uden hurtige løsninger aktiveres.
\end{itemize}

\begin{center}\rule{0.5\linewidth}{0.5pt}\end{center}

\section{Fase 5: Systemfejl og
nødprocedurer}\label{fase-5-systemfejl-og-nuxf8dprocedurer}

\begin{itemize}
\tightlist
\item
  Situationen er indtruffet uden varsel.
\item
  Nødløsninger iværksættes, f.eks. papirbaserede skemaer og
  kommunikation uden cloud-tjenester.
\item
  Målet er at opretholde minimumsdrift og genetablere stabilitet
  hurtigst muligt.
\item
  Plan for genetablering af systemer igangsættes, når situationen
  tillader det.
\end{itemize}

\newpage

\chapter{Fase 1: Observation og Forberedende
Overvågning}\label{fase-1-observation-og-forberedende-overvuxe5gning}

\section{Fasedefinition}\label{fasedefinition}

Fase 1 repræsenterer det laveste trusselsniveau. Her vurderes risikoen
for tab af adgang til amerikansk software, tjenester eller licenser som
\textbf{meget usandsynlig}. Det betyder, at der \textbf{ikke igangsættes
nogen tekniske eller organisatoriske ændringer}. Formålet med denne fase
er at sikre opmærksomhed, skabe en bevidsthed om potentielle
afhængigheder og opretholde en basal informationsberedskab i
organisationen.

\section{Overordnede handlinger}\label{overordnede-handlinger}

\subsection{Dokumentation af nuværende
infrastruktur}\label{dokumentation-af-nuvuxe6rende-infrastruktur}

\begin{itemize}
\item
  Hvis der ikke foreligger en velstruktureret og opdateret dokumentation
  over infrastrukturen, bør der oprettes et grundlæggende overblik med
  følgende elementer:

  \begin{itemize}
  \tightlist
  \item
    Hardware (servere, klienter, netværksudstyr, mobile enheder)
  \item
    Software (installerede systemer og applikationer, versioner og
    konfigurationer)
  \item
    Licenser og abonnementer (inkl. udløbsdatoer og
    leverandørbetingelser)
  \item
    Certifikater og adgangskoder (opbevares sikkert og adskilt, fx i
    password manager)
  \item
    Eksterne vedligeholdelsespartnere og serviceaftaler
  \end{itemize}
\item
  Identificere kontaktpersoner, der er ansvarlige for beredskabsplanens
  videreførelse, hvis trusselsniveauet ændrer sig
\item
  Sikre opbevaring og synlighed af eksisterende beredskabsplaner (fx
  fase 2--5)
\item
  Orientere relevante ledelsesniveauer om eksistensen af
  beredskabsmodellen
\item
  Vurdere behovet for fremtidig kommunikation eller informationsstrategi
\item
  Følge relevante nyhedsstrømme, trusselsrapporter eller politiske
  signaler der kan påvirke IT-afhængigheder
\end{itemize}

\section{Kommunikation}\label{kommunikation}

\begin{itemize}
\tightlist
\item
  Ingen bred intern eller ekstern kommunikation
\item
  Informationsniveau begrænses til beredskabsansvarlige og evt.
  topledelse
\item
  Forberede skabeloner til fremtidig eskalationskommunikation
\end{itemize}

\section{Output fra denne fase}\label{output-fra-denne-fase}

\begin{itemize}
\tightlist
\item
  Navngivne kontaktpersoner og ansvarlige
\item
  Journalisering af beslutning om at forblive i fase 1
\item
  Notat om næste vurderingstidspunkt (f.eks. hvert kvartal eller ved
  specifik geopolitisk ændring)
\end{itemize}

\section{Risikostyring i denne fase}\label{risikostyring-i-denne-fase}

\begin{itemize}
\tightlist
\item
  Lav trussel: Ingen aktioner iværksættes
\item
  Overvågning og bevidsthed prioriteres over teknisk forberedelse
\item
  Næste fase (Fase 2) aktiveres ved ændret politisk, teknisk eller
  kommerciel situation
\end{itemize}

\newpage

\chapter{Fase 2: Planlægningsfase}\label{fase-2-planluxe6gningsfase}

\section{Fasedefinition}\label{fasedefinition}

I denne fase er truslen om licens- eller tjenesteafbrydelse stadig
usandsynlig, men vurderes som tilstrækkeligt realistisk til, at der bør
udarbejdes kontingensplaner og tekniske overvejelser. Formålet er at
dokumentere og analysere vores afhængighed af amerikansk-baserede
teknologier og forberede strategier til at kunne håndtere en potentiel
afbrydelse.

\section{Overordnede handlinger}\label{overordnede-handlinger}

\subsection{System- og
tjenesteidentifikation}\label{system--og-tjenesteidentifikation}

\begin{itemize}
\tightlist
\item
  Identificere alle kritiske systemer og tjenester afhængige af
  amerikanske leverandører (Microsoft, Apple, Google, Cisco m.fl.)
\end{itemize}

\subsection{Juridisk og teknisk
vurdering}\label{juridisk-og-teknisk-vurdering}

\begin{itemize}
\tightlist
\item
  Vurdere juridiske og tekniske muligheder for erstatning eller
  afkobling
\end{itemize}

\subsection{Udkast til
migreringsplaner}\label{udkast-til-migreringsplaner}

\begin{itemize}
\item
  Udarbejde udkast til migreringsplaner for (inklusive øvrige
  afhængigheder, værktøjer og systemnære komponenter):

  \begin{itemize}
  \item
    Operativsystemer
  \item
    Kontorpakker
  \item
    E-mail og kalender
  \item
    Autentificering og adgangsstyring
  \item
    Cloud-lagring og samarbejdsværktøjer
  \item
    Netværksudstyr, telefoni og mobile løsninger (inkl. vurdering af
    afhængighed af Apple og Android enheder og mulige alternativer som
    Fairphone, /e/OS eller Linux-baserede systemer)
  \item
    Backup-løsninger og lagring
  \item
    Databaser
  \item
    DNS
  \item
    Firmwareopdateringer (BIOS, UEFI, drivere og anden enhedsnær
    software med afhængighed af producentens infrastruktur)
  \item
    Øvrige specialiserede applikationer afhængigt af organisationens
    behov, f.eks.:

    \begin{itemize}
    \tightlist
    \item
      AI-værktøjer
    \item
      Grafisk software
    \item
      Videoredigering
    \item
      Overvågning og adgangskontrol
    \item
      CAD-software
    \item
      Social media-værktøjer (EU-venlige alternativer)
    \end{itemize}
  \end{itemize}
\end{itemize}

\section{Teknologiske alternativer
(eksempler)}\label{teknologiske-alternativer-eksempler}

\begin{quote}
\textbf{Bemærkning:} Enkelte af de foreslåede løsninger har oprindelse i
USA (f.eks. Keycloak, FreeIPA, Jitsi, GitLab CE, Ansible, Puppet), men
er fuldt open source og kan selvhostes uden afhængighed af kommercielle
tjenester eller licensbindinger. De er medtaget i kraft af deres
robusthed, modenhed og uafhængighed i driftsmiljøer. Målet er ikke
fuldstændig isolation, men reduktion af kritisk afhængighed.
\end{quote}

\begin{itemize}
\tightlist
\item
  \textbf{Operativsystem:} Linux Mint, Debian eller Arch i testmiljø
\item
  \textbf{Kontorpakke:} LibreOffice, OnlyOffice eller Collabora
\item
  \textbf{E-mail:} Dovecot + Postfix, evt. i kombination med Nextcloud
  Kalender og Kontakter
\item
  \textbf{AD/Azure AD:} Keycloak og FreeIPA
\item
  \textbf{Cloud/SharePoint:} Nextcloud
\item
  \textbf{Kommunikation (Teams):} Jitsi og Matrix
\item
  \textbf{GitHub:} Gitea eller selvhostet GitLab CE
\item
  \textbf{SCCM:} Ansible, shell-scripts eller Puppet
\item
  \textbf{ERP/Navision:} Odoo, Dolibarr, Tryton
\item
  \textbf{Database:} PostgreSQL, MariaDB, MySQL (alle open source og
  EU-venlige)
\item
  \textbf{Backup-løsninger:} BorgBackup, Restic, Duplicity, eller Veeam
  med lokal lagring
\item
  \textbf{Mobile enheder:} Fairphone med /e/OS eller Ubuntu Touch,
  PinePhone, Librem 5
\end{itemize}

\section{Kommunikation}\label{kommunikation}

Etablering af en klar kommunikationsplan er central i denne fase og skal
sikre, at information om planen, dens formål og forløb bliver
kommunikeret effektivt og rettidigt til alle interessenter.

Kommunikationen bør:

\begin{itemize}
\tightlist
\item
  Udformes som en plan for internt og eksternt beredskab
\item
  Understøtte gennemsigtighed og tillid til planens formål og metode
\item
  Muliggøre videndeling og samarbejde med andre organisationer og
  open-source-miljøer
\item
  Formidle opdateringer løbende gennem passende kanaler
\item
  Publicering af planen i open source-format (f.eks. via GitHub), med
  henblik på at styrke gennemsigtighed og tillid til planens formål og
  metode. Dette vil samtidig gøre det muligt for andre organisationer at
  kommentere, genbruge og bidrage til indholdet på tværs af sektorer og
  landegrænser.
\item
  Starte kontakt til andre organisationer og open-source-miljøer for
  videndeling og fælles udvikling
\end{itemize}

\section{Arbejdsgruppe}\label{arbejdsgruppe}

Det anbefales at oprette en tværfaglig arbejdsgruppe med repræsentanter
fra relevante dele af organisationen. Denne gruppe skal understøtte
koordinering, forankring og fremdrift i planens arbejde.

Arbejdsgruppen bør:

\begin{itemize}
\item
  Bestå af:

  \begin{itemize}
  \tightlist
  \item
    Styregruppe
  \item
    Økonomiansvarlige
  \item
    Implementeringsansvarlige
  \item
    Supportfunktioner
  \item
    Kompetenceudviklingsansvarlige
  \item
    Ansvarlige for intern kommunikation
  \end{itemize}
\item
  Have ansvar for:

  \begin{itemize}
  \tightlist
  \item
    Løbende afstemning med ledelse og nøglepersoner
  \item
    Indsamling af feedback fra brugere og medarbejdere
  \item
    Vurdering af behov for eksternt samarbejde og koordinering
  \item
    Forberedelse af kulturskift og organisatorisk forankring
  \end{itemize}
\end{itemize}

\section{Output fra denne fase}\label{output-fra-denne-fase}

\subsection{Dokumenterede
kontingensplaner}\label{dokumenterede-kontingensplaner}

\begin{itemize}
\item
  Dokumenterede kontingensplaner for alle hovedsystemer, herunder for
  hver løsning:

  \begin{itemize}
  \tightlist
  \item
    Valg af alternativ og begrundelse
  \item
    Implementeringskrav (teknisk og organisatorisk)
  \item
    Migrationsmuligheder og -metoder (eksportfunktioner, scripting,
    manuel overførsel eller nyopsætning fra bunden)
  \item
    IT-kompetenceudvikling: krav til opkvalificering eller behov for
    outsourcing
  \item
    Slutbruger-kompetenceudvikling: træning og support
  \item
    Budgetovervejelser: estimater for licens, hardware og drift
  \item
    Tidsforbrug og estimeret implementeringshorisont
  \end{itemize}
\end{itemize}

\subsection{Pilotprojekter}\label{pilotprojekter}

\begin{itemize}
\item
  Pilotprojekter defineret og eventuelt igangsat, herunder:

  \begin{itemize}
  \item
    Opsætning af sandkassemiljø til test af FOSS-løsninger uden
    produktionseksponering
  \item
    For hvert pilotprojekt bør følgende defineres:

    \begin{itemize}
    \tightlist
    \item
      Hvad måles og testes?
    \item
      Hvilke konkrete spørgsmål ønskes besvaret?
    \item
      Hvor længe kører forsøget?
    \item
      Hvilke systemer og brugergrupper indgår?
    \item
      Hvem evaluerer og dokumenterer erfaringerne?
    \end{itemize}
  \end{itemize}
\end{itemize}

\subsection{Licens- og
afhængighedskortlægning}\label{licens--og-afhuxe6ngighedskortluxe6gning}

\begin{itemize}
\item
  Klarlæggelse af licensforhold og afhængigheder, herunder:

  \begin{itemize}
  \tightlist
  \item
    Hvilke systemer kræver løbende licensvalidering?
  \item
    Hvilke systemer påvirkes indirekte, hvis licens eller adgang til et
    andet system ophører?
  \item
    Er der afhængigheder til cloud-baserede login-, aktiverings- eller
    valideringstjenester?
  \item
    Hvor hurtigt forventes forskellige systemer at stoppe med at fungere
    ved licensafbrydelse?
  \item
    Hvilke systemer har nødlicens, offline-funktionalitet eller lokale
    fallback-muligheder?
  \end{itemize}
\end{itemize}

\subsection{Kompetenceudvikling}\label{kompetenceudvikling}

\begin{itemize}
\tightlist
\item
  Overblik over nødvendige kompetenceudviklingsområder (f.eks. Linux,
  Ansible, e-mailadministration)
\end{itemize}

\section{Risikostyring i forhold til fokus i denne
fase}\label{risikostyring-i-forhold-til-fokus-i-denne-fase}

\begin{itemize}
\tightlist
\item
  Ingen akutte handlinger tages endnu
\item
  Fokus på dokumentation, analyse og samarbejde
\item
  Planen evalueres og opdateres løbende ved ændring i trusselsbilledet
\end{itemize}

\newpage

\chapter{Fase 3: Blød overgang}\label{fase-3-bluxf8d-overgang}

\section{Fasedefinition}\label{fasedefinition}

I denne fase vurderes truslen som \textbf{realistisk og tiltagende}, men
der er stadig tid til at foretage en planlagt og kontrolleret overgang.
Der fokuseres på en \textbf{gradvis og økonomisk afbalanceret
transition}, der tager hensyn til organisationens forandringsparathed og
ressourcer.

Målet er at reducere kritiske afhængigheder og samtidig sikre, at
forandringer implementeres med så lav forstyrrelse af driften som
muligt.

\section{Overordnede handlinger}\label{overordnede-handlinger}

\begin{itemize}
\item
  Prioritere kortlægning og migrering af \textbf{kritiske systemer og
  hardware} først, hvor afhængighed udgør en væsentlig forretningsrisiko
\item
  Udarbejde en prioriteringsmatrix for systemer og enheder baseret på:

  \begin{itemize}
  \tightlist
  \item
    Kritikalitet for forretningsdrift
  \item
    Risiko ved fortsat afhængighed af amerikanske leverandører
  \item
    Kompleksitet og omkostninger ved migrering
  \end{itemize}
\item
  Prioritere migrering af systemer med lav kompleksitet og høj risiko
  ved fortsat afhængighed
\item
  Evaluere og justere pilotprojekter, der er igangsat i Fase 2, og
  begynde udrulning i begrænsede driftsmiljøer
\item
  Balancere økonomiske hensyn med nødvendigheden af at sikre
  uafhængighed
\item
  Indgå eventuelle nye aftaler med EU-baserede leverandører og
  open-source communities
\end{itemize}

\section{Pilotprojekter og gradvis
implementering}\label{pilotprojekter-og-gradvis-implementering}

\begin{itemize}
\item
  Gennemføre pilotprojekter for eks.:

  \begin{itemize}
  \tightlist
  \item
    Linux på klientmaskiner i udvalgte afdelinger
  \item
    Alternativ kontorpakke (LibreOffice, OnlyOffice eller Collabora)
  \item
    Lokal e-mailserver (Dovecot + Postfix)
  \item
    Udskiftning af Teams med Matrix eller Jitsi
  \item
    GitHub-migrering til Gitea eller selvhostet GitLab CE
  \end{itemize}
\item
  For hver pilot skal følgende være defineret:

  \begin{itemize}
  \tightlist
  \item
    Succeskriterier for drift og brugertilfredshed
  \item
    Plan for opfølgning og evaluering
  \item
    Økonomisk ramme og tidsplan for eventuel udvidelse
  \end{itemize}
\end{itemize}

\section{Kommunikation}\label{kommunikation}

\begin{itemize}
\tightlist
\item
  Tydelig kommunikation til hele organisationen om overgangen
\item
  Løbende informationsmøder og statusopdateringer
\item
  Synliggørelse af resultater og erfaringer fra pilotprojekter
\item
  Forventningsafstemning med brugere om ændringer i arbejdsgange og
  værktøjer
\end{itemize}

\section{Kompetenceudvikling}\label{kompetenceudvikling}

\begin{itemize}
\item
  Identificere konkrete kompetencebehov i IT-afdelingen i forbindelse
  med overgangen
\item
  Definere hvilke kompetencer der kan opkvalificeres internt, og hvor
  der er behov for ekstern træning eller konsulentbistand
\item
  Udarbejde en træningsplan med:

  \begin{itemize}
  \tightlist
  \item
    Tidshorisont for opkvalificering
  \item
    Økonomisk budget for kompetenceudvikling
  \item
    Ansvarlige for gennemførelse af træning
  \end{itemize}
\end{itemize}

\section{Arbejdsgruppe og ledelse}\label{arbejdsgruppe-og-ledelse}

\begin{itemize}
\tightlist
\item
  Udvide arbejdsgruppen fra Fase 2 med nøglepersoner fra drift og
  support
\item
  Sikre ledelsesmæssig opbakning til ændringer
\item
  Udarbejde detaljerede implementeringsplaner og milepæle
\end{itemize}

\section{Output fra denne fase}\label{output-fra-denne-fase}

\subsection{Dokumenterede beslutninger og
milepæle}\label{dokumenterede-beslutninger-og-milepuxe6le}

\begin{itemize}
\tightlist
\item
  Gennemførte og evaluerede pilotprojekter
\item
  Migrering af mindre kritiske systemer
\item
  Klar og synlig kommunikationsplan for næste skridt
\item
  Beslutning om, hvilke områder der er klar til fuld overgang,
\end{itemize}

\subsection{Næste skridt}\label{nuxe6ste-skridt}

\begin{itemize}
\tightlist
\item
  Beslutning om eventuel eskalering til Fase 4
\item
  Planlægning af fuld migrering for prioriterede områder
\item
  Kommunikation af næste beslutningspunkt til hele organisationen
\end{itemize}

\section{Risikostyring i forhold til fokus i denne
fase}\label{risikostyring-i-forhold-til-fokus-i-denne-fase}

\begin{itemize}
\tightlist
\item
  Aktiv overvågning af trusselsbilledet
\item
  Løbende vurdering af om tempoet i overgangen skal øges
\item
  Risikoanalyse for hver større ændring, der implementeres
\item
  Udarbejdelse af fallback-planer i tilfælde af uforudsete udfordringer
\end{itemize}

\newpage

\chapter{Fase 4: Akut overgang}\label{fase-4-akut-overgang}

\section{Fasedefinition}\label{fasedefinition}

I denne fase vurderes truslen som \textbf{umiddelbar og troværdig}.
Situationen kræver øjeblikkelig handling for at sikre fortsat drift og
mindske risikoen for systemnedbrud eller tab af adgang til essentielle
tjenester.

Alle tilgængelige ressourcer mobiliseres for hurtigst muligt at
implementere de løsninger, der er forberedt i tidligere faser. Fokus er
på at sikre, at kritiske systemer fungerer og er uafhængige af sårbare
eksterne afhængigheder.

\section{Overordnede handlinger}\label{overordnede-handlinger}

\begin{itemize}
\tightlist
\item
  Indhendt info fra Fase 2 og 3
\item
  Prioritere migrering af \textbf{kritiske tjenester} under tidspres
\item
  Allokere al tilgængelig IT- og supportkapacitet til opgaven
\item
  Aktivere nødprocedurer for områder, hvor migrering ikke kan
  gennemføres i tide
\item
  Suspendere ikke-kritiske projekter for at frigive ressourcer
\item
  Ekspresanskaffelse af nødvendigt hardware eller software
\item
  Etablere hurtige beslutningsveje og ledelsesmæssig eskalation
\end{itemize}

\section{Kritiske fokusområder}\label{kritiske-fokusomruxe5der}

\begin{itemize}
\tightlist
\item
  Migrering af e-mail og kommunikation til sikre og kontrollerede
  løsninger
\item
  Sikring af adgangs- og identitetssystemer (f.eks. Keycloak, FreeIPA)
\item
  Implementering af lokale fil- og backup-løsninger
\item
  Opsætning af interne DNS- og netværkstjenester
\item
  Udskiftning af cloud-baserede samarbejdsværktøjer med lokale
  alternativer (f.eks. Nextcloud)
\end{itemize}

\section{Kommunikation}\label{kommunikation}

\begin{itemize}
\tightlist
\item
  Kriseinformationsplan aktiveres
\item
  Tæt kommunikation mellem IT, ledelse og berørte afdelinger
\item
  Kort og klar kommunikation til alle brugere om ændringer og
  begrænsninger
\item
  Brug af alternative kommunikationskanaler, hvis cloud-baserede
  løsninger fejler
\end{itemize}

\section{Kompetenceudvikling}\label{kompetenceudvikling}

\begin{itemize}
\tightlist
\item
  Fokus på akut oplæring i de alternative løsninger, der nu
  implementeres
\item
  Udnytte interne superbrugere som støtte til hurtig vidensoverførsel
\item
  Afholde korte, fokuserede workshops eller online sessioner for brugere
  og IT-personale
\end{itemize}

\section{Arbejdsgruppe og ledelse}\label{arbejdsgruppe-og-ledelse}

\begin{itemize}
\item
  Der åbnes op for anvendelse af overarbejde og ekstra ressourcer i
  denne fase. Ledelsen har samtidig ansvar for at overvåge
  medarbejdernes belastningsniveau og aktivt forebygge udbrændthed.
\item
  Arbejdsgruppen fra Fase 2 aktiveres som den centrale operative enhed i
  denne fase.
\item
  Arbejdsgruppen udvides med:

  \begin{itemize}
  \tightlist
  \item
    Direktøren eller en ledelsesrepræsentant med fuldt mandat til at
    træffe hurtige og afgørende beslutninger.
  \item
    Økonomiansvarlige og ressourcestyringsansvarlige for hurtig
    godkendelse af nødvendige anskaffelser og allokering af midler.
  \end{itemize}
\item
  Arbejdsgruppen har ansvaret for den praktiske eksekvering, mens
  ledelsesrepræsentanter sikrer hurtig beslutningstagen og fjernelse af
  organisatoriske barrierer.
\item
  Sikre konstant ledelsesopbakning og tilstedeværelse i den operative
  drift.
\item
  Etablere hurtige og effektive beslutningsveje for at kunne reagere på
  nye udviklinger i realtid.
\end{itemize}

\section{Output fra denne fase}\label{output-fra-denne-fase}

\begin{itemize}
\tightlist
\item
  Kritiske systemer migreret til sikre og bæredygtige løsninger
\item
  Nødprocedurer aktiveret for områder, hvor migrering ikke kunne ske
\item
  Kommunikationskanaler etableret og aktivt anvendt
\item
  Beredskabsstatus løbende dokumenteret og rapporteret til ledelsen
\end{itemize}

\section{Risikostyring i forhold til fokus i denne
fase}\label{risikostyring-i-forhold-til-fokus-i-denne-fase}

\begin{itemize}
\tightlist
\item
  Kontinuerlig overvågning af trusselsbilledet
\item
  Daglige evalueringer af fremdrift og kapacitetsbehov
\item
  Løbende risikovurdering af prioriterede migrationsopgaver
\item
  Udarbejdelse af handlings planer for overgangen til Fase 5, hvis
  situationen forværres yderligere
\end{itemize}

\newpage

\chapter{Fase 5: Systemfejl og
Nødprocedurer}\label{fase-5-systemfejl-og-nuxf8dprocedurer}

\section{Fasedefinition}\label{fasedefinition}

Situationen er \textbf{indtruffet uden varsel}. Kritiske systemer er
utilgængelige, og adgangen til essentielle IT-tjenester er mistet. Der
er ikke længere tid eller mulighed for at gennemføre planlagte
migreringer. Fokus er nu på at sikre absolut minimumsdrift og stabilitet
gennem nødprocedurer, indtil normale forhold gradvist kan genetableres.

\section{Overordnede handlinger}\label{overordnede-handlinger}

\begin{itemize}
\tightlist
\item
  Udarbejde en prioriteret liste over kritiske driftsfunktioner, som
  skal opretholdes under alle omstændigheder
\item
  Aktivere evt forberedte nødprocedurer fra fase 4
\item
  Tage manuelle, papirbaserede arbejdsgange i brug (skemaer,
  tilstedeværelseslister, manuel registrering af kritiske data)
\item
  Etablere lokal kommunikation via ikke-digitale midler eller
  alternative netværk (radio, SMS, fysisk mødeaktivitet)
\item
  Udpege et fysisk samlingspunkt for koordinering og kommunikation
\item
  Udpege nøglepersoner med ansvar for hver kritisk driftsfunktion
\item
  Sikre fysisk adgang til nødvendige faciliteter og udstyr uden
  afhængighed af digitale adgangssystemer
\end{itemize}

\section{Kritiske fokusområder}\label{kritiske-fokusomruxe5der}

\begin{itemize}
\tightlist
\item
  Bevare informationssikkerhed trods manuelle arbejdsgange
\item
  Beskytte tilgængeligt data og udstyr mod fysisk tab eller skader
\item
  Prioritere ressourcer til de mest nødvendige funktioner (kerneopgaven,
  sikkerhed, lønudbetaling)
\item
  Sørge for fysisk tilstedeværelse af ledelses- og
  kommunikationsansvarlige
\item
  Vurdere muligheder for hurtigst muligt at genetablere
  kerne-IT-funktioner lokalt uden eksterne afhængigheder
\end{itemize}

\section{Kommunikation}\label{kommunikation}

\begin{itemize}
\tightlist
\item
  Anvende nødkommunikationsplanen baseret på manuelle og offline-kanaler
\item
  Sikre daglige briefinger med statusopdateringer til medarbejdere og
  interessenter
\item
  Udpege en kommunikationsansvarlig til at koordinere og formidle al
  intern og ekstern kommunikation
\item
  Dokumentere alle midlertidige foranstaltninger og beslutninger for
  senere analyse og evaluering
\end{itemize}

\section{Kompetenceudvikling og
support}\label{kompetenceudvikling-og-support}

\begin{itemize}
\tightlist
\item
  Sikre adgang til krisepsykolog eller stresshåndteringsressourcer, hvis
  krisen trækker ud
\item
  Udnytte eksisterende superbrugere og personale med erfaring i manuelle
  processer
\item
  Aktivere backup-ressourcer, herunder pensionerede IT-medarbejdere
  eller studiejobsøgende med relevant erfaring, hvis muligt. Overvej
  også at inddrage medarbejdere fra andre afdelinger til at assistere
  med ikke-komplekse opgaver efter en kort grundtræning.
\item
  Fokusere på rolig kriseledelse og støtte til medarbejdere for at undgå
  panik og forvirring
\end{itemize}

\section{Arbejdsgruppe og ledelse}\label{arbejdsgruppe-og-ledelse}

\begin{itemize}
\tightlist
\item
  Arbejdsgruppen omdannes til en krisestab med fuldt mandat til at
  træffe hurtige beslutninger
\item
  Overarbejde kan aktiveres for at opretholde kritiske funktioner.
  Ledelsen har ansvar for at balancere brugen af overarbejde med
  hensynet til medarbejdernes trivsel og forebyggelse af udbrændthed.
  Der bør etableres klare rammer for, hvor længe overarbejde kan
  pålægges uden at iværksætte personalerotation eller inddragelse af
  backup-ressourcer.
\item
  Direktøren eller stedfortræder er til stede fysisk som kriseleder
\item
  Alle beslutninger og tiltag skal logges manuelt for senere opfølgning
  og rapportering
\end{itemize}

\section{Output fra denne fase}\label{output-fra-denne-fase}

\subsection{Stabilisering og Tilpasning efter
Krise}\label{stabilisering-og-tilpasning-efter-krise}

\begin{itemize}
\item
  Udarbejde en plan for systematisk dokumentation af nye
  arbejdsprocesser, så erfaringerne fra krisen fastholdes og bliver en
  integreret del af den fremtidige drift.
\item
  Arbejdsgruppen har ansvaret for at lede og koordinere denne fase
  baseret på organisationens behov og de erfaringer, der er gjort under
  krisen.
\item
  Udarbejde en plan for, hvordan alle medarbejdere løbende får den
  nødvendige oplæring til at tilpasse sig de nye systemer og
  arbejdsprocesser.
\item
  Vurdere behov for at skabe eller tilpasse eksisterende
  arbejdsprocesser i alle afdelinger, så de understøtter de
  implementerede systemer bedst muligt.
\item
  HR i samarbejde med afdelingsledere skal udarbejde en plan for
  håndtering af overarbejde efter krisen, som sikrer:

  \begin{itemize}
  \tightlist
  \item
    At medarbejdere tilbydes mulighed for afspadsering eller
    kompensation.
  \item
    At udbrændthed forebygges gennem planlagt restitution.
  \item
    At for mange medarbejdere ikke holder fri samtidig, så
    organisationens drift kan fortsætte uforstyrret.
  \end{itemize}
\item
  Udarbejdelse af plan for digitalisering og korrekt integration af data
  indsamlet manuelt under krisesituationen
\item
  Udpege ansvarlig for sikring og opbevaring af krisedokumentation til
  brug i audit og senere evaluering
\item
  Opnået opretholdelse af minimumsdrift via nødprocedurer
\item
  Nødløsninger dokumenteret og erfaringer opsamlet
\item
  Plan for genetablering af systemdrift udarbejdet og igangsat, når
  situationen tillader det
\end{itemize}

\section{Risikostyring i forhold til fokus i denne
fase}\label{risikostyring-i-forhold-til-fokus-i-denne-fase}

\begin{itemize}
\tightlist
\item
  Udarbejde klare kriterier for afslutning af nødprocedurer og
  overgangen til normaliseret drift
\item
  Udarbejde en plan for vedvarende krisedrift, herunder
  personalerotation, forsyningssikkerhed og fortsat opretholdelse af
  kerneopgaven
\item
  Kontinuerlig vurdering af personalets trivsel og belastning
\item
  Overvågning af fysisk sikkerhed og adgangsforhold
\item
  Løbende vurdering af, hvornår overgangen til genetableringsfase kan
  påbegyndes
\item
  Evaluering og prioritering af hvilke systemer, der genetableres først
\end{itemize}

\newpage

\chapter{Licens og Deling}\label{licens-og-deling}

\begin{quote}
\textbf{Åben anvendelse tilladt:} Denne plan stilles frit til rådighed
for enhver organisation eller enkeltperson, der ønsker at bruge,
tilpasse eller videreudvikle den. Det er ikke nødvendigt med tilladelse
eller kreditering, men feedback, forbedringer samt forslag til
samarbejde og fælles videreudvikling modtages meget gerne.
\end{quote}

Dette dokument er tænkt som et bidrag til fælles resiliens i en usikker
digital verden. Vi tror på, at gennemsigtighed, samarbejde og deling er
centrale byggesten i opbygningen af digital suverænitet. Derfor lægger
vi ikke begrænsninger på brugen, og vi byder både offentlige og private
aktører velkommen til at anvende materialet.

\subsection{Anbefalinger ved genbrug:}\label{anbefalinger-ved-genbrug}

\begin{itemize}
\tightlist
\item
  Tilpas indholdet til egen kontekst og behov.
\item
  Del gerne forbedringer eller tilpasninger tilbage til fællesskabet.
\item
  Undgå at gengive dokumentet som officiel myndighedsvejledning uden
  lokal tilpasning.
\end{itemize}

\subsection{Licens}\label{licens}

Denne plan stilles til rådighed under \textbf{Creative Commons CC0 1.0
Universal (Public Domain Dedication)}.

\begin{itemize}
\item
  \textbf{CC0 betyder:}

  \begin{itemize}
  \tightlist
  \item
    Du må frit kopiere, ændre, tilpasse og distribuere materialet til
    ethvert formål, også kommercielt.
  \item
    Der kræves ingen kreditering af ophav eller kilde.
  \end{itemize}
\end{itemize}


\end{document}